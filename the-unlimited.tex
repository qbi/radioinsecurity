\documentclass[12pt]{book}
\usepackage[T1]{fontenc}
\usepackage[utf8]{inputenc}
\usepackage[ngerman]{babel}

\begin{document}

\title{The Unlimited}
%\author{Unknown}
%\date{2017}
\maketitle

\chapter{Speicher}

\section{Virtuelle Speicherverwaltung (32-Bit)}

In der virtuelen Speicherverwaltung benutzt man für Prozesse das Flat Memory
Model, auch bekannt als Protected Mode. Die Zusammenarbeit zwischen dem
Prozessor und dem Betriebssystemkern ermöglichen es, dass jedem Prozess einen
Adressraum von 4 GB (32-Bit) zur Verfügung steht. Die virtuelle Speicherverwaltung
arbeitet nicht mehr kombiniert mit einer Segment- und einer Offsetadresse, wie
es unter DOS üblich war, sondern man benutzt eine einzige 32-Bit große
(Offset-)Adresse für den kompletten Adressraum.

Der Systemkern setzt beim Start einer Anwendung einen 4 GB umfassenden
Adressraum, auch wenn gar keine 4 GB RAM zur Verfügung stehen. Dabei belegt
eine Anwendung nur so viel Speicher, wie der Programmcode und die zugehörigen
Daten benötigen. Ist noch zusätzlichen Speicher nötig, so kann
dieser, dank der virtuellen Speicherverwaltung, nachträglich angefordet werden.
Verwaltet wird die virtuelle Speicherverwaltung durch den Systemkern und basiert
auf einem grundlegenden Mechanismus der Speicheradressierung des Prozessors.

So ist oft nicht der gesamte virtuelle Speicher einer Anwendung im
Arbeitsspeicher verfügbar, da sich eine Anwendung in der Regel nur in Teilen des
Codes oder der Daten bewegt. Dieser Umstand macht es Möglich, das man Teile des
Arbeitsspeichers auf einer Festplatte auslagern kann. Zugriffe auf ausgelagerten
Speicher werden vom Prozessor abgefangen und ausgelagerte Teile werden von ihm
erst wieder in den Arbeitsspeicher geladen. Die Anwendung selbst bekommt von
diesem Vorgang nichts mit. Damit das Ein- und Auslagern möglichst effizient ist,
teilt der Prozessor den virtuellen Speicher in 4 KB große Abschnitte, sogenannte
Pages, auf.

Wann immer Maschinencode auf einen Adressraum zugreifen will, berechnet der
Prozessor zuerst einmal aus der Adresse die Nummer der Page. Er bedient sich dazu
der Page-Tabellen auch bekannt als Page-Directorys. Über sie wird der
Adressraum Stück für Stück auf Pages im pyhsischen Speicher verteilt. Der
Prozessor gibt zwar das Format dieser Tabellen vor, aber verwaltet werden sie
vom Betriebssystem. Der Eintrag der Page-Tabelle informiert den Prozessor auch
darüber, ob sich eine Page im physischen Speicher befindet oder ob diese
eventuell auf eine Festplatte ausgelagert wurde. Ein Eintrag in einer
Page-Tabelle enthält für diesen Zweck verschiedene Flags, die unter anderem
definieren, auf welche Art auf die Pages zugegriffen werden darf (read-only, …).
Sollte Speicher ausgelagert sein, so wird dies dem Systemkern gemeldet, welcher
sich dann um das Nachladen der Pages kümmert. Sollte es passieren das der
komplette physische Speicher in Benutzung ist, so müssen erst andere Pages
ausgelagert werden.

Was ein Programm als durchgängigen Adressraum sieht, verteilt sich so in
Wirklichkeit über viele Pages, welche an ganz unterschiedlichen Stellen im
physischen Speicher existieren. Der Speicher einer Anwendung kann bis zu
Unkenntlichkeit fragmentiert im physischen Speicher vorliegen, doch es wird ihr
verborgen bleiben, weil Sie keinen Speicherzugriff ausführen kann, ohne das der
Prozessor dies bemerken würde. Da manche ausgelagerte Daten eventuell zeitnah
wieder benötigt werden, versucht das Betriebssystem dies zu erahnen und lädt
manche Daten schon wieder in den Speicher, bevor versucht wurde auf diese
zuzugreifen. Zu beachten ist auch, dass nicht der komplette Adressraum von 4 GB
einer Anwendung zur Verfügung steht, sondern Pages von z. B. Bibliotheken
können in mehreren Anwendungen verwendet werden. Somit ist auch nicht zu
vergessen, dass eine Instanz einer Bibliothek pro Prozess auch einen
Datenbereich benötigt.

Abschließend sei erwähnt das Adressen größer 3 GB dem Systemkern vorbehalten
sind, somit ist anzunehmen, dass wenn die CPU in diesen Adressraum wechselt, sie
auch von Ring-3 zu Ring-0 wechselt.

\section{Paging}

Paging ist ein integraler Bestandteil des Protected Mode und seit dem 80386
verfügbar. Für den Einsatz des Paging-Mechanismus ist das PG-Bit im
Control-Register 0 (CR0) des Prozessors verantwortlich. Dieses Bit steht auf 0
nach dem das System im Real-Mode startet, hier werden lineare Speicheradressen
direkt auf physikalische Adressen abgebildet und Paging findet noch nicht statt.
Schalte das Betriebssystem den Prozessor in den Protected-Mode, so setzt es
dieses Bit auf 1. Jede Adresse, die der Prozessor bezüglich eines
Maschinenbefehls verarbeitet, wird in diesem Modus erst einer Page zugeordnet.
Ein Speicherzugriff wird ab jetzt auf eine Page umgeleitet. Die Größe einer Page
beträgt 4 kB und jede Page beginnt an einer durch 4 kB teilbaren physikalischen
Adresse. Der Adressraum wird dadurch in 2hoch20 verschiedene Pages aufgeteilt, die
jeweils 2hoch12 Byte (4 kB) umfassen. Die Ausrichtung auf 4 kB ermöglicht, dass die
untere 12 Bitgrenze dafür sorgt, dass lineare Adressen und physikalische
Adressen identisch sind. Sie stellen eine Art Offset der Page da. Die oberen 20
Bit der linearen Adresse werden für die Nummerierung der Page verwendet. Diese
20 Bit werden isoliert und als Index in der Page-Table verwendet, aus welcher
die physikalische Adresse entnommen wird. Die Einträge in der Page-Table sind 32
Bit breit, es werden aber nur 20 Bit benötigt, denn sie muss an einer durch 4 kB
teilbaren physikalischen Speicherstelle beginnen. Egal welche Page man nimmt,
die unteren 12 Bit des Index sind deshalb immer 0. Somit werden dort die
verschiedenen Flags untergebracht, welche z. B. definieren, ob eine Page
ausgelagert ist. Damit für eine komplette Page-Table keine kompletten 4 MB
verwendet werden müssen, wurde der Mechanismus von Intel etwas verfeinert.
Anstelle einer großen Page-Table für den gesamten 32-Bit-Adressraum wurde eine
zweistufige Organisation mit mehreren Page-Tables gewählt. Ausgangspunkt ist das
Page-Directory, über das mehrere kleinere Page-Tables verwaltet werden. Es
werden die oberen 10 Bit der linearen Adresse als Index in Page-Directory
verstanden, welches aus 1024 Einträgen besteht. Eine Adresse des Page-Directory
wird über das CR3-Register bereitgestellt. Die einzelnen Einträge einer
Page-Directory enthalten Zeiger mit den Adressen der verschiedenen Page-Tables.
Die Nummer des Eintrags, welcher für die Umrechnung der Adresse herangezogen
wird, ergibt sich aus dem Besitz 12–21 der linearen Adresse. Eine lineare
Adresse wird so in 2 Teile aufgespalten. Das Page-Directory und jede
Page-Tabelle nimmt so die Adressen von 1024 Pages auf. Jeder Eintrag in der
Page-Directory deckt so einen Bereich von 4 MB innerhalb des linearen
Adressraums ab. Das System kann so die Page-Tables nach Bedarf anlegen.
Besonders wichtig für die Zusammenarbeit mit dem Betriebssystem sind die unteren
20 Bit der Page-Table-Einträge. Hier werden wie schon erwähnt die Flags der
Pages untergebracht. Besonders wichtig ist das Present-Flag, es muss vom
Betriebssystem auf 0 gestellt werden, wenn eine Page ausgelagert wurde. Für den
Prozessor ist, dass das Signal um eine Exception auszulösen. Dadurch erlangt das
Betriebssystem die Kontrolle über die Programmausführung und erhält die
Möglichkeit die Page nachzuladen und den Page-Table-Eintrag mit einer neuen
Basisadresse der Page im Speicher zu initialisieren. Das Acccess-Flag wird
gesetzt, wenn auf eine Page das erste Mal zugegriffen wurde. So kann das
Betriebssystem später besser entscheiden, welche Pages zuerst ausgelagert werden
sollen, insofern der physikalische Speicher knapp wird. Hier werden die
bevorzugt, wo das Access-Flag noch auf 0 steht. Pages, welche einen
Schreibzugriff hatten, bei denen wird das Dirty-Bit auf 1 gesetzt. Es ist
einfacher Pages zu laden, welche unverändert sind. Andere Flags realisieren ein
Schutzmechanismus zwischen System-Code und User-Code oder markieren eine Page
als schreibgeschützt. Bei Missachtung dieser Flags wird eine Exception
ausgelöst.

\section{Dynamischer Speicher}

Anwendungen können über verschiedene Möglichkeiten, während der Ausführung
Speicher allozieren. Hierfür verwendet man in der Regel die Funktionen der
entsprechenden Bibliotheken, welche die Speicheranforderung dieser Funktionen
aus dem virtuellen Speicher bedienen. Es ist möglich Speicher zu allozieren,
welcher sich direkt in das Paging-Konstrukt eingefügt. Diese Art wird oft
benutzt, um Dateien in den Speicher einzulesen oder große Arrays anzulegen. Es
ist so möglich sich regelrecht Abschnitte im linearen Adressraum zu reservieren.
Das System nutzt diese Möglichkeit, um Teile des Adressraums für sich und den
geladenen Prozess zu reservieren. Auch der User-Code kann davon profitieren,
überall dort, wo Datenstrukturen während der Programmausführung dynamisch
wachsen. Dies ergibt Sinn, da der Erweiterungsspeicher beim Vergrößern einer
Struktur nicht einfach hinten angehangen werden kann. Andernfalls würde ein
höherer Aufwand mit Zeigern und verknüpften Listen betrieben werden oder man
müsste Speicher beim Vergrößern wandern lassen. Allozierter virtuelle Speicher
des User-Codes steht somit dem Systemkern oder verwendeten Bibliotheken nicht
zur Verfügung. Benötigt man Speicher für Bäume und Listen, so kann man über die
Heap-Funktionen Speicher Byte-genau allozieren. Um dem Problem zu entgehen, das
große reservierte Speicherabschnitte den physikalischen Speicher oder die
Auslagerungsdatei regelrecht sprengen. Ein Prozess muss den Speicher, welchen er
wirklich nutzen will noch commiten. Das hat den Vorteil, dass Arrays trotzdem
sequenziell wachsen können, obwohl hier erst mal nur ein kleiner Teil des
reservierten Speichers benutzt. Ab hier wird der dynamische Speicher im
Page-Konstrukt wirklich angelegt. Ist der committete Bereich für die anfallenden
Daten nicht groß genug, so wird der Prozessor eine Exception auslösen.

\section{Heap}

Wenn man viele kleine Speicherblöcke verwalten möchte, ohne sich die Frage
stellen zu müssen, ob dieser Speicher verfügbar ist, dann benutzt man die
verschiedenen Heap-Funktionen. Die C-Standard-Funktionen hierzu sind zum
Beispiel malloc(), realloc() und free(), welche auf dem entsprechenden
Betriebssystem-API basieren. Über das Betriebssystem-API ist es in der Regel
möglich, mehrere Heaps mit unterschiedlichen Größen anzulegen. Der Speicher für
Heaps wird aus dem virtuellen Adressraum der Anwendung bezogen. Die jeweiligen
Heaps werden dann über ein Handle angesprochen. Dem Prozess-Heap gibt es bereits
beim Start eines jeden Prozesses. Der Loader legt in ihm Informationen ab, die
der Systemkern verwalten muss. Der angefragte Speicher wird immer auf eine
Page-Größe aufgerundet.

\section{Code-/Textsegment}

Ein Prozess ist grundlegend in 3 Regionen aufgebaut. Diese Regionen sind das
Codesegment, das Datensegment und der Stack. Diese Segmente ergeben sich in der
Regel direkt durch das Parsen einer ausführbaren Datei. Das Codesegment, welches
oft auch als Textsegment betitelt wird, beinhaltet, wie der Name schon sagt, den
auszuführenden Code und ist schreibgeschützt. Ein Versuch, hier Daten zu ändern,
würde zu einer Speicherzugriffsverletzung führen. Das Codesegment beginnt in der
Regel im niedrigen Adressraum.


\section{Datensegment}

Im Datensegment sind initialisierte und nicht initialisierte globale Variablen
angesiedelt, sowie Konstanten. Auch der dynamische Speicher und der Heap werden
hier hinzugezählt, diese Bereiche ergeben sich aber nicht aus der ausführbaren
Datei.

\section{Stack}

Der Stack befindet sich am Ende des Prozessadressraumes und wächst in Richtung
Code und Datensegment. Der Stack kann mit den Prozessorbefehlen PUSH und POP be-
und entladen werden. Sein Aufbau ähnelt hier einem Stapel von Tellern. Der Stack
arbeitet nach dem Last-In-First-Out-Prinzip (LIFO). So kann mit POP nur das
Element entladen werden, welches zuletzt auf dem Stack abgelegt wurde. Der Stack
wird in der Regel benutzt, um Parameter und Rückgabewerte zu übergeben und
diverse Rücksprungsadressen zu vorherigen Prozeduren vorzuhalten. Üblich ist
auch das die lokalen Variablen einer Prozedur, auf dem Stack angelegt werden,
somit ist es möglich verschiedene Prozeduren verschieden oft aufzurufen. Der
Stackzeiger (SP) ist ein Register ähnlich wie der Instruktionszeiger und zeigt
immer auf das oberste Element im Stack. Das Ende des Stacks ist eine feste
Adresse und für gewöhnlich auch das Ende des kompletten Programms. In
Ausnahmefällen gibt es auch Stackimplementierungen, wo der Stack zu einer
größeren Adresse hin wächst, bei gängigen Prozessoren ist dies dennoch nicht der
Fall. Als Zusatz zu dem Stackzeiger benutzt man oft noch den Base Pointer (BP),
welcher auf irgend einen Wert im Stack zeigen kann. Zusätzlich wird der
Framezeiger (FP) benutzt, welche auch in vielen Texten als Local Base Pointer
(LB) betitelt wird. Der Base Pointer wird benutzt um lokale Variablen über einen
Offset zu referenzieren. Dieser Offset beginnt bei der Adresse, welche im SP
gespeichert ist. Bei diversen PUSHs und POPs kann sich dieser Offset aber
verändern. In der Regel wirft der Compiler ein Auge hierauf und korrigiert die
entsprechenden Werte. Viele Compiler benutzen aber das zweite Register, den
Framezeiger und referenzieren damit lokale Variablen sowie Parameter, da der
Wert im Framezeiger nicht abhängig vom Stackzeiger ist. Der Zugriff auf
Variablen sowie Parameter wird hier nun mit einem Offset zu dem Wert im
Framezeiger ermöglicht. Das erste was eine Funktion bzw. Prozedur tun muss,
nachdem sie aufgerufen wurde, sie speichert den aktuellen Wert des "alten"
Framezeiger auch auf dem Stack und kopiert die aktuelle Adresse des Stackzeiger
in den Framezeiger, danach wird der Stackzeiger verändert, um Platz für neue
lokale Variablen zu schaffen. Dieser Platz, ist der berüchtigte Puffer, aus dem
man ausbrechen könnte, um eine Rücksprungadresse zu verändern. Soll eine
Prozedur beendet werden, so muss der vorherige Zustand wieder genau hergestellt
werden, Prozessoren bringen hier die Instruktionen ENTER und LEAVE bzw. LINK und
UNLINK mit, natürlich kann man hier auch einfach wieder mit POP und verändern
der Adresse den Urzustand wieder herstellen. Diese beiden Vorgehensweisen werden
auch als Prozedur-Prolog bzw. -Epilog verstanden. Das gesamte Vorgehen kann aber
bei verschiedenen Prozessoren etwas abweichen. Sollte der Platz zwischen
Datensegment und Stack einmal zu gering werden, so wird der Prozess angehalten
und kurz darauf mit zusätzlichem Speicher fortgesetzt.

\chapter{Speicherüberlauf}

\section{Stack-Based Buffer Overflow}

Bei vielen C Implementationen ist es möglich den vorhandenen Stack zu
korrumpierten, indem man über das Ende eines im Stack existierenden Puffers
schreibt. Ein Puffer ist ein Bereich im Speicher, welcher nacheinander
verschiedene Instanzen desselben Datentypen beinhaltet. In C sind dies für
gewöhnlich char-Arrays um Strings zu speichern. Globale Variablen werden beim
Laden der Anwendung im Datensegment angesiedelt. Lokale Variablen in Funktionen
werden aber dynamisch im Stack angelegt. Verlässt man den Puffer, so
überschreiben weitere Daten die Werte im Stack, welche sich hinter diesem Puffer
befinden. Somit ist es möglich die Rücksprungsadressen von z. B. Prozeduren zu
verändern und dafür zu sorgen, dass der Instruktionszeiger an einer anderen
Stelle den Code fortführt. In Assembler bzw. eigentlich eher im Objektcode ist
es möglich Prozeduren bzw. Funktionen über ein CALL anzuspringen und über einen
RET wieder zurückzukehren. Ein CALL tut dabei nichts anderes, als die Adresse
nach dem CALL als Rücksprungsadresse auf dem Stack abzulegen und das Register
des Instruktionszeigers auf die Anfangsadresse der Prozedur zu setzen, ähnlich
wie bei einem MOV. Ein direktes Ändern des Instruktionszeigers ist MOV aber
nicht erlaubt. Um zum ursprünglichen Code zurück zu gelangen, entfernt RET die
Rücksprungsadresse vom Stack und ändert den Instruktionszeiger auf genau
diese. Nun passiert es oft, dass Quelltexte so geschrieben worden, dass Strings
von dem ein Puffer in einen anderen Puffer byteweise kopiert werden, so lange
bis ein Stringendezeichen (der Wert 0) in ein Byte des Quellstring vorkommt.
Ist der Zielpuffer z. B. aber nur 16 Byte groß, die Quelle aber z. B. 256 Byte
groß, so würde die schon etwas obsolete Funktion strcpy (die Alternative ist
strncpy), aus der C-Standardbibliothek, erbarmungslos bis zum Ende kopieren.
Da hier vermutlich auf Maschinencodeebene immer nur ein Zeiger inkrementiert
wird und mit einem MOV die Daten kopiert werden. Passiert zur Laufzeit ein
solcher Fehler, so bekommt man oft, ein segmentation fault. Die Adresse, welche
in der Quellzeigervariable steht, wird mit Bytes überschrieben und somit
überschreibt dieser Fehler sogar die Quelladresse von strcpy, danach versucht
MOV, Daten von einer Adresse zu lesen, die nicht zur Verfügung steht. Hier sollte
angemerkt sein das MOV in verschiedenen Varianten vorliegt und auch in der Lage
ist words (16-Bit), doublewords (32-Bit) und quadwords (64-Bit) zu kopieren.
Findet man nun solch einen Fehler in ein Programm, so kann man durch die
richtige Anzahl von Bytes, so viel vom Stack überschreiben, das nur die
Rücksprungsadresse betroffen ist und so kein Speicherzugriffsfehler ausgelöst
wird. Die Bytes, welche wir hier von strcpy kopieren lassen, enthalten natürlich
schon unseren Code, welcher später ausgeführt werden soll.

\section{Global-Based Buffer Overflow}

\section{Heap-Based Buffer Overflow}

\section{Stackoverflow}
%https://blog.fefe.de/?ts=a7b7880f

\section{Integeroverflow}

\section{Shellcode (Payload)}

\chapter{Gegenmaßnahmen}

\section{NX-Bit}

\subsection{Angriffe NX-Bit}

\section{ASLR}

\subsection{KASLR}

\subsection{KARL}
%https://heise.de/-3767821

\subsection{PIE-Flag}

\subsection{Angriffe ASLR}

\subsubsection{Spraying}

\section{Stack-Cookie}

\section{CFI}

\section{Stack Smashing Protector (ehemals ProPolice)}

\section{Stack Guard}

\section{Syscall-Filtering (https://en.wikipedia.org/wiki/Seccomp)}

\section{Sandboxing/Container (Chroot, LXC, Jails, Apps, Browser, Java, ...)}

\subsection{Jailbreak}

\chapter{Finden von Overflows}

\chapter{Vermeiden von Overflows}

\end{document}
