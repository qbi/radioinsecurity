\documentclass[12pt]{book}
\usepackage[T1]{fontenc}
\usepackage[utf8]{inputenc}
\usepackage[ngerman]{babel}

\begin{document}

\title{The Unlimited}
%\author{Unknown}
%\date{2017}
\maketitle

\chapter{Speicher}

\section{Virtuelle Speicherverwaltung (32-Bit)}

In der virtuellen Speicherverwaltung benutzt man für Prozesse das Flat Memory
Model, auch bekannt als Protected Mode. Die Zusammenarbeit zwischen dem
Prozessor und dem Betriebssystemkern ermöglichen es, dass jedem Prozess einen
Adressraum von 4 GB (32-Bit) zur Verfügung steht. Die virtuelle
Speicherverwaltung arbeitet nicht mehr kombiniert mit einer Segment- und einer
Offsetadresse, wie es unter DOS üblich war, sondern man benutzt eine einzige
32-Bit große (Offset-)Adresse für den kompletten Adressraum.

Der Kernel setzt beim Start einer Anwendung einen 4 GB umfassenden
Adressraum, auch wenn gar keine 4 GB RAM zur Verfügung stehen. Dabei belegt
eine Anwendung nur so viel Speicher, wie der Programmcode und die zugehörigen
Daten benötigen. Ist noch zusätzlichen Speicher nötig, so kann
dieser, dank der virtuellen Speicherverwaltung, nachträglich angefordert werden.
Verwaltet wird die virtuelle Speicherverwaltung durch den Kernel und basiert
auf einem grundlegenden Mechanismus der Speicheradressierung des Prozessors.

So ist oft nicht der gesamte virtuelle Speicher einer Anwendung im
Arbeitsspeicher verfügbar, da sich eine Anwendung in der Regel nur in Teilen des
Codes oder der Daten bewegt. Dieser Umstand macht es Möglich, das man Teile des
Arbeitsspeichers auf einer Festplatte auslagern kann. Zugriffe auf ausgelagerten
Speicher werden vom Prozessor abgefangen und ausgelagerte Teile werden von ihm
erst wieder in den Arbeitsspeicher geladen. Die Anwendung selbst bekommt von
diesem Vorgang nichts mit. Damit das Ein- und Auslagern möglichst effizient ist,
teilt der Prozessor den virtuellen Speicher in 4 KB große Abschnitte, sogenannte
Pages, auf.

Wann immer Maschinencode auf einen Adressraum zugreifen will, berechnet der
Prozessor zuerst einmal aus der Adresse die Nummer der Page. Er bedient sich
dazu der Page-Tabellen auch bekannt als Page-Directorys. Über sie wird der
Adressraum Stück für Stück auf Pages im physischen Speicher verteilt. Der
Prozessor gibt zwar das Format dieser Tabellen vor, aber verwaltet werden sie
vom Betriebssystem. Der Eintrag der Page-Tabelle informiert den Prozessor auch
darüber, ob sich eine Page im physischen Speicher befindet oder ob diese
eventuell auf eine Festplatte ausgelagert wurde. Ein Eintrag in einer
Page-Tabelle enthält für diesen Zweck verschiedene Flags, die unter anderem
definieren, auf welche Art auf die Pages zugegriffen werden darf (Read-Only, …).
Sollte Speicher ausgelagert sein, so wird dies dem Kernel gemeldet, welcher
sich dann um das Nachladen der Pages kümmert. Sollte es passieren das der
komplette physische Speicher in Benutzung ist, so müssen erst andere Pages
ausgelagert werden.

Was ein Programm als durchgängigen Adressraum sieht, verteilt sich so in
Wirklichkeit über viele Pages, welche an ganz unterschiedlichen Stellen im
physischen Speicher existieren. Der Speicher einer Anwendung kann bis zu
Unkenntlichkeit fragmentiert im physischen Speicher vorliegen, doch es wird ihr
verborgen bleiben, weil Sie keinen Speicherzugriff ausführen kann, ohne das der
Prozessor dies bemerken würde. Da manche ausgelagerte Daten eventuell zeitnah
wieder benötigt werden, versucht das Betriebssystem dies zu erahnen und lädt
manche Daten schon wieder in den Speicher, bevor versucht wurde auf diese
zuzugreifen. Zu beachten ist auch, dass nicht der komplette Adressraum von 4 GB
einer Anwendung zur Verfügung steht, sondern Pages von z. B. Bibliotheken
können in mehreren Anwendungen verwendet werden. Somit ist auch nicht zu
vergessen, dass eine Instanz einer Bibliothek pro Prozess auch einen
Datenbereich benötigt.

Abschließend sei erwähnt das Adressen größer 3 GB dem Kernel vorbehalten sind.
Wenn die CPU in diesen Adressraum wechselt, dann wechselt sie auch oft von einem
in den anderen Ring. In den verschieden Ringen stehen der CPU unterschiedlich
ausgeprägte Befehlssätze zur Verfügung. So kann z. B. ein User-Prozess in Ring 3
nicht auf Hardware zugreifen, der Kernel in Ring 0 aber sehr wohl. Die CPU
ermittelt den Ring, in welchem sie sich gerade befindet, über die
Supervisor-Bits im Status-Register.

\section{Paging}

Paging ist ein integraler Bestandteil des Protected Mode und seit dem 80386
verfügbar. Für den Einsatz des Paging-Mechanismus ist das PG-Bit im
Control-Register 0 (CR0) des Prozessors verantwortlich. Dieses Bit steht auf 0
nach dem das System im Real-Mode startet, im Real-Mode werden lineare
Speicheradressen direkt auf physikalische Adressen abgebildet. Paging findet
hier noch nicht statt. Schalte das Betriebssystem den Prozessor in den
Protected-Mode, so setzt er dieses Bit auf 1. Jede Adresse, die der Prozessor
bezüglich eines Maschinenbefehls verarbeitet, wird jetzt einer Page zugeordnet.

Die Größe einer Page beträgt 4 kB (4096 Byte) und jede Page beginnt an einer
durch 4 kB teilbaren physischen Adresse. Der Adressraum wird dadurch in 1048576
(20-Bit) verschiedene Pages aufgeteilt, die jeweils 4 kB (12-Bit) umfassen. Die
Ausrichtung auf 4 kB ermöglicht es, dass die untere 12 Bitgrenze dafür sorgt,
dass lineare Adressen und physikalische Adressen identisch sind, sie stellen
eine Art Offset einer Page da. Die oberen 20-Bit der linearen Adresse werden
für die Nummerierung der Pages verwendet. Diese 20-Bit werden isoliert und als
Index in der Page-Table verwendet, aus welcher die physischen Adresse entnommen
wird.

Die Einträge in der Page-Table sind 32-Bit breit, es werden aber nur 20-Bit
benötigt, denn sie muss, wie bereits erwähnt, an einer durch 4 kB teilbaren
physischen Speicherstelle beginnen. Egal welche Page man nimmt, die unteren
12-Bit des Index sind deshalb eigentlich immer 0. Um diese 12-Bit aber doch
sinnvoll zu nutzen, werden dort verschiedenen Flags untergebracht, welche z. B.
Definieren, ob eine Page ausgelagert ist oder als Read-Only deklariert wurde.

Damit für eine komplette Page-Table keine kompletten 4 MB verwendet werden
müssen, wurde der Mechanismus von Intel etwas verfeinert.  Anstelle einer
großen Page-Table für den gesamten 32-Bit-Adressraum wurde eine zweistufige
Organisation mit mehreren Page-Tables gewählt. Ausgangspunkt ist das
Page-Directory, über das mehrere kleinere Page-Tables verwaltet werden. Es
werden die oberen 10-Bit der linearen Adresse als Index im Page-Directory
verstanden, welches aus 1024 Einträgen besteht. Eine Adresse des Page-Directory
wird über das CR3-Register bereitgestellt. Die einzelnen Einträge einer
Page-Directory enthalten Zeiger mit den Adressen der eigentlichen Page-Tables.
Die Nummer des Eintrags, welcher für die Umrechnung der Adresse herangezogen
wird, ergibt sich aus den Bits 12–21 der linearen Adresse. Die lineare Adresse
wird so in 2 Teile aufgespalten, Directory und Table.

Das Page-Directory und jede Page-Table nimmt so die Adressen von 1024 Einträgen
auf. Jeder Eintrag in der Page-Directory deckt so einen Bereich von 4 MB
(4096x1024) innerhalb des linearen Adressraums ab. Der Vorteil ist, dass das
System so die Page-Tables nach Bedarf anlegen kann.

Besonders wichtig für die Zusammenarbeit mit dem Betriebssystem sind die
unteren 20-Bit der Page-Table-Einträge. Hier werden wie schon erwähnt die Flags
der Pages untergebracht. Besonders wichtig ist das Present-Flag, es muss vom
Betriebssystem auf 0 gestellt werden, wenn eine Page ausgelagert wurde. Für den
Prozessor ist, dass das Signal um eine Exception auszulösen. Dadurch erlangt
das Betriebssystem die Kontrolle über die Programmausführung und erhält die
Möglichkeit die Page nachzuladen und den Page-Table-Eintrag mit einer neuen
Basisadresse der Page im Speicher zu initialisieren.

Wird auf eine Page das erste Mal zugegriffen, so wird im Zuge der
Aktualisierung der Basisadresse, das Access-Flag auf 1 gesetzt.  So kann das
Betriebssystem später besser entscheiden, welche Pages, sollte der Speicher mal
knapp werden, zuerst ausgelagert werden. Hier werden die bevorzugt, wo das
Access-Flag noch auf 0 steht. Pages, welche einen Schreibzugriff hatten, bei
denen wird zudem das Dirty-Bit auf 1 gesetzt, weil es einfacher ist Pages zu
laden, welche noch unverändert sind.

Andere Flags realisieren ein Schutzmechanismus zwischen System-Code und
User-Code oder markieren eine Page, wie oben erwähnt, als schreibgeschützt. Bei
Missachtung dieser Flags, wird eine Exception im Prozessor ausgelöst.

\section{Prozessspeicher}

Ein Prozess ist grundlegend in 3 Regionen aufgebaut. Diese Regionen sind das
Codesegment, das Datensegment und der Stack. Diese Segmente ergeben sich in der
Regel direkt durch das Parsen einer ausführbaren Datei.

\subsection{Textsegment}

Das Textsegment beinhaltet das Codesegment für den auszuführenden Code und
Kostanten. Konstanten sind Daten, welche zur Laufzeit des Programms nicht
verändert werden dürfen. Das komplette Textsegment ist schreibgeschützt. Ein
Versuch, hier Daten zu ändern, würde zu einer Speicherzugriffsverletzung führen.
Das Textsegment beginnt in der Regel im niedrigen Adressraum.

\subsection{Datensegment}

Im Datensegment sind initialisierte und nicht initialisierte globale Variablen
angesiedelt. Der dynamische Speicher wird hier auch hinzugezählt, dieser Bereich
ergibt sich aber nicht durch das Parsen einer ausführbaren Datei sonders erst
zur Laufzeit.

\subsubsection{Dynamischer Speicher}

Anwendungen können über verschiedene Möglichkeiten, während der Ausführung
Speicher allozieren. Hierfür verwendet man in der Regel die Funktionen der
entsprechenden Bibliotheken, welche die Speicheranforderung dieser Funktionen
aus dem virtuellen Speicher bedienen.

Es ist möglich Speicher zu allozieren, welcher sich direkt in das
Paging-Konstrukt eingefügt. Diese Art wird oft benutzt, um Dateien in den
Speicher einzulesen oder große Arrays anzulegen. Es ist so möglich sich
regelrecht Abschnitte im linearen Adressraum zu reservieren. Das
Betriebssystem nutzt diese Möglichkeit, um Teile des Adressraums für sich und
den geladenen Prozess zu reservieren. Auch der User-Code kann davon
profitieren, überall dort, wo Datenstrukturen während der Programmausführung
dynamisch wachsen. Dies ergibt Sinn, da der Erweiterungsspeicher beim
Vergrößern einer Datenstruktur, wie z. B. einer verketteten Liste, nicht
einfach hinten an gehangen werden kann. Andernfalls würde ein höherer Aufwand
mit Zeigern und verknüpften Listen betrieben werden oder man müsste Speicher
beim Vergrößern wandern lassen. Allozierter virtuelle Speicher des User-Codes
steht somit dem Kernel oder verwendeten Bibliotheken nicht zur Verfügung.
Benötigt man Speicher für Bäume und Listen, so kann man über die
Heap-Funktionen Speicher Byte-genau allozieren. So kann man dem Problem
entgehen, das große reservierte Speicherabschnitte den physikalischen Speicher
oder die Auslagerungsdatei regelrecht sprengen.

Ein Prozess muss den Speicher, welchen er wirklich nutzen will noch commiten.
Das hat den Vorteil, dass Arrays trotzdem sequenziell wachsen können, obwohl
hier erst mal nur ein kleiner Teil des reservierten Speichers benutzt wird.
Nach dem Commit wird der dynamische Speicher im Page-Konstrukt wirklich
angelegt. Sollte der committete Bereich für die anfallenden Daten nicht groß
genug sein, so wird der Prozessor eine Exception auslösen.

\paragraph{Heap}

steht für Halde oder Haufen und wird benötigt, wenn man viele kleine
Speicherblöcke verwalten möchte, ohne sich die Frage stellen zu müssen, ob
dieser Speicher verfügbar ist. Die C-Standardbibliothek stellt hier die
Funktionen malloc(), realloc() und free() bereit, welche auf dem entsprechenden
Betriebssystem-API basieren. Über das Betriebssystem-API ist es in der Regel
möglich, mehrere Heaps mit unterschiedlichen Größen anzulegen. Der Speicher für
Heaps wird aus dem virtuellen Adressraum der Anwendung bezogen. Die jeweiligen
Heaps werden dann über ein Handle (Zeiger) angesprochen. Den Prozess-Heap gibt
es bereits beim Start eines jeden Prozesses, da der Loader in ihm Informationen
ablegt, die der Kernel verwalten muss. Der angefragte Speicher wird immer
auf eine Page-Größe aufgerundet.

\subsection{Stack}

Der Stack befindet sich am Ende des Prozessadressraumes und wächst in Richtung
Code und Datensegment. Der Stack kann mit den Prozessor- bzw. Assemblerbefehlen
PUSH und POP be- und entladen werden. Das heißt sein Aufbau ähnelt hier einem
Stapel von Tellern, welchen der Stack nach dem Last-In-First-Out-Prinzip (LIFO)
abarbeitet. So kann mit POP nur das Element entladen werden, welches zuletzt
auf dem Stack abgelegt wurde.

Der Stack wird in der Regel benutzt, um Parameter und Rückgabewerte zu übergeben
und diverse Rücksprungadressen zu vorherigen Prozeduren vorzuhalten.  Üblich ist
auch das die lokalen Variablen einer Prozedur (Funktion), auf dem Stack angelegt
werden. Das vorhalten der lokalen Variablen auf dem Stack ermöglicht es, dass
verschiedene Prozeduren verschieden oft aufgerufen werden können. Der Bereich im
Stack, indem alle relevanten Daten einer Prozedur enthalten sind, wird als Stack
Frame bezeichnet. Wurden mehrere Prozeduren aufgerufen, so existieren auch
mehrere dieser Frames im Stack, für jeden Aufruf eine eigene.

Der Stack Pointer (SP) ist ein Register, ähnlich dem Instruction Pointer (IP)
und zeigt immer auf die Adresse des obersten Elements im Stack. Das Ende des
Stacks ist eine feste Adresse und für gewöhnlich auch das Ende des kompletten
Programms. Es gibt auch Ausnahmefälle in der Stackimplementierungen, hier wächst
der Stack dann zu einer größeren Adresse hin. Bei gängigen Prozessoren ist dies
dennoch nicht der Fall. Als Zusatz, um eine Variable oder Parameter
referenzieren zu können, braucht der Stack Pointer noch den Base Pointer (BP),
welcher auf irgend einen Wert im Stack zeigen kann. Auch als Frame Pointer (FP)
kommt er hier ggf. zum Einsatz, welcher auch in vielen Texten als Local Base
Pointer (LB) bezeichnet wird.

Der Base Pointer wird benutzt um lokale Variablen über einen Offset zu
referenzieren. Dieser Offset beginnt bei der Adresse, welche im Stack Pointer
gespeichert ist. Bei weiteren Prozeduraufrufen oder einfach nur weiteren
PUSH-Anweisungen wird der Offset aber quasi größer und muss verändert werden,
diese Veränderungen werden in der Regel vom Compiler vorgenommen. Muss der
Compiler die Offsetadresse im Base Pointer oft aktualisieren, ist das je nach
dem ein nicht zu unterschätzender Aufwand, da bei zu großen Offsets ggf. auch
noch weitere Code ausgeführt werden muss. In Manchen Fällen ist der Compiler gar
nicht in der Lage diese Offset-Korrekturen fehlerfrei vorzunehmen. Eine andere
und oft genutzte Strategie um lokale Variablen zu referenzieren ist hier die
Benutzung des Base Pointers als Frame Pointers. Der Zugriff auf Variablen und
Parameter wird bei dieser Vorgehensweise mit einem Offset zu dem Wert im Frame
Pointer ermöglicht, da der Frame Pointer nicht abhängig von dem Wert im Stack
Pointer ist, muss auch der Compiler hier keine nachträglichen Korrekturen
vornehmen. Der Wert im Frame Pointer ist so gewählt, dass für gewöhnlich
Parameter, Rücksprungadresse und der alte Frame Pointer einen positiven Offset
besitzen und lockale Variablen einen negativen.

Das erste was eine Prozedur tun muss, nachdem sie aufgerufen wurde, ist sie
speichert den aktuellen Wert des Frame Pointers auf dem Stack und kopiert die
aktuelle Stack Pointer Adresse in den Frame Pointer. Darauffolgend wird der
Stack Pointer so verändert, dass genug Platz für die lokale Variablen entsteht,
unser sogenannter Stack Frame wird hier angelegt. Oft fällt die größe des Stack
Frames etwas größer aus als die Summe der Bytes der lokalen Variablen, bei einer
32-Bit Architektur ist ein Element des Stacks genau 4 Bytes groß. Das hat den
Sinn, das Werte in die Register der CPU passend abgelegt werden können und nicht
zusätzlicher Code erzeugt werden muss, der z. B. Werte zweier Variablen in einem
Register erkennen und verwalten muss.

Soll eine Prozedur beendet werden, so muss der vorherige Zustand natürlich
wieder genau hergestellt werden, Prozessoren bringen hier die Instruktionen
ENTER und LEAVE bzw. LINK und UNLINK mit, bzw. kann man hier auch einfach
wieder mit POP und verändern der Adressen den Urzustand herstellen.

Diese Vorgehensweisen werden als Prozedur-Prolog bzw. -Epilog verstanden und
können bei verschiedenen Prozessoren etwas abweichen.

\chapter{Speicherüberlauf}

\section{Stack-Based Buffer Overflow}

Bei vielen C Implementationen ist es möglich den vorhandenen Stack zu
korrumpierten, indem man über das Ende eines im Stack existierenden Puffers
schreibt. Ein Puffer ist ein Bereich im Speicher, welcher nacheinander
verschiedene Instanzen derselben Datentypen beinhaltet. In C sind dies für
gewöhnlich char-Arrays um Strings zu speichern.

Man muss hier unterscheiden zwischen Globale Variablen, welche beim Laden der
Anwendung im Datensegment angelegt werden und lokale Variablen aus Funktionen
die dynamisch auf dem Stack erzeugt werden.

Verlässt man diesen sogenannten Puffer, so überschreiben die folgenden Daten
Werte im Stack, welche sich hinter dem Puffer befinden. Somit ist es möglich,
die Rücksprungsadressen von Prozeduren zu verändern und dafür zu sorgen, dass
der Instruction Pointer an einer anderen Stelle den Code fortführt.

In Assembler bzw. eigentlich eher im Objektcode werden Prozeduren bzw.
Funktionen über ein CALL aufgerufen und mit einem RET ist man dann in der Lage
zur vorherigen zurück zu springen. Ein CALL tut dabei nichts anderes, als die
Adresse des Befehls, welcher sequentiell quasi direkt nach dem CALL ausgeführt
würde als Rücksprungsadresse auf dem Stack abzulegen und das Register des
Instruction Pointer auf die Anfangsadresse der neuen Prozedur zu setzen, ähnlich
wie mit einem MOV. Ein direktes Ändern des Instruction Pointer ist MOV aber
nicht erlaubt. Um zum ursprünglichen Code zurück zu gelangen, entfernt RET den
aktuellen Stackframe inklusive der Rücksprungsadresse vom Stack und ändert dabei
den Instruction Pointer wieder auf genau diese Adresse.

Nun passiert es oft, dass Quelltexte so geschrieben werden, dass Strings von dem
einen Puffer in einen anderen Puffer byteweise kopiert werden, so lange bis ein
Stringendezeichen (der Wert 0) in einem Byte des Quellstring vorkommt. Ist der
Zielpuffer z. B. aber nur 16 Byte groß, die Quelle aber 256 Byte groß, so würde
die veraltete Funktion strcpy (die Alternative ist strncpy), aus der
C-Standardbibliothek, erbarmungslos bis zum Ende des Quellstrings alle Bytes
kopieren. Auf Maschinencodeebene wird hier vermutlich nur ein Register
inkrementiert oder dekrementiert und mit einem MOV kopiert man dann schrittweise
die Daten. Hier sollte angemerkt sein das MOV in verschiedenen Varianten
vorliegt und auch in der Lage ist Words (16-Bit), Doublewords (32-Bit) und
Quadwords (64-Bit) zu kopieren.

Passiert zur Laufzeit ein solcher Fehler, so bekommt man oft, ein Segmentation
Fault. Die Rücksprungadresse, welche RET dann benutzt, wurde hier vermutlich mit
einem Wert überschreiben der nicht existiert.

\section{Global-Based Buffer Overflow}

\section{Heap-Based Buffer Overflow}

\section{Stackoverflow}
%https://blog.fefe.de/?ts=a7b7880f
Sollte der Platz zwischen Datensegment und Stack einmal zu gering werden, so
wird der Prozess angehalten und kurz darauf mit zusätzlichem Speicher
fortgesetzt.

\section{Integeroverflow}

\section{Shell Code (Payload)}

Als Shell Code bezeichnet man die Instruktionen, welche man als fremden Code
ausführt, nachdem man die Rücksprungsadresse einer Prozedur verändert hat. In
den meisten Fällen möchte man mit diesem Code eine Shell, im besten Fall von
außen (remote), verfügbar machen. Über eine Shell hat man dann die Möglichkeit
beliebige andere Programme zu starten und diverse Einstellungen zu setzen. Da
die meisten Programme keinen Code mitbringen, der eine Shell startet und man zu
diesem hin springen könnte, so muss man eigenen Code platzieren.

Eigenen Code platziert man, indem man diesen in den Puffer schreibt, über dessen
Grenzen wir hinaus schreiben um die Rücksprungsadresse zu verändern. Das
bedeutet dass die Rücksprungsadresse genau auf das erste Segment des Puffers
zeigen sollte.

Code, welcher ein anderes Programm startet, wie z.B. eine Shell, wird in C
vermutlich mit einer der exec Systemfunktionen realisiert. Systemfunktionen auch
bekannt als Systemaufrufe (systemcalls/syscalls) werden in der Regel über
Softwareinterrupts (INT) oder andere spezielle CPU-Befehle (SYSENTER/SYSCALL)
realisiert. Ein Softwareinterrupt (auch Exception genannt) unterbricht die
Programmausführung im Benutzer-Modus (Ring-3) und erzwingt das Ausführen eines
Exception-Handlers im Kernel-Modus (Ring-0). In Ring-0 steht der komplette
Befehlssatz und der gesamte Speicherbereich der CPU zur Verfügung. Aus
Sicherheitgründen laufen normale Prozesse deswegen in Ring-3, wo nur ein Teil
des Befehlssatzes und des Speichers (die Verwaltung des virtuellen Speichers
kann nur in Ring-0 vorgenommen werden) zur Verfügung stehen. Die Nutzung von
Privilegierungsebenen ist sinnvoll, um die Hardware zu abstrahieren und um
Prozesse voneinander abzuschotten zu können. In CPU und ggf. MMU müssen
Schaltungen bestehen, die bei jedem Befehl bzw. Speicherzugriff prüfen, ob
dieser im aktuellen Ring erlaubt ist. Falls ein Prozess etwas nicht Erlaubtes
durchführen möchte, wird er unterbrochen und eine Betriebssystem-Routine
aufgerufen, deren Aufgabe es ist, entsprechend zu reagieren. In welchem Ring,
welcher Code ausgeführt wird, definiert man über die Tabellen, worüber die
Speicherseiten verwaltet werden.

Systemaufrufe sind eine Sammlung von Codeeinheiten im Kernel, die man benutzen
kann und um verschiedene Dinge zu tun, wie z. B. der Zugriff auf Dateien, das
erzeugen von Signalen oder auch das Starten anderer Programme. Hier findet ein
Kontextwechsel statt, da ein normaler Prozess, wie oben schon erwähnt, nicht
selbst auf Hardware zugreifen darf. Das bedeutet, dass man den bedroffenden
Prozess unterbricht und die Kontrolle der CPU dem Kernel übergeben wird. Bevor
eine dieser Codeeinheiten im Kernel ausgeführt wird, überprüft dieser in der
Regel ob ein Prozess bzw. Benutzer auch dazu berechtigt ist. Der Kernel
beherbergt eine Liste aller ihm bekannten Systemaufrufe, die so genannte System
Call Table, welche als Sprungtabelle genutzt wird. Jedem Systemaufruf wird dort
eine eindeutige Nummer zugeordnet.

Um einen Systemaufruf durchzuführen, wird die Nummer des gewünschten Aufrufs in
das EAX-Register der CPU geschrieben und anschließend der zuständige
Softwareinterrupt, unter Linux ist das in der Regel 80h, über seine Nummer
ausgelöst. Für Argumente an den Systemaufruf, aber auch Rückgabewerte, benutzt
man hier in der Regel auch die Register der CPU.

Somit ist klar, dass uns eine Fülle an brauchbaren Funktionen zur Verfügung
steht, welche wir mit wenigen Prozessorbefehlen ausführen können. Durch
Systemaufrufe werden wir in die Lage versetzt, eine Shell zu starten, indem wir
einfach eine Pfadangabe auf die Shellbinärdatei als String im Speicher ablegen
und die Startadresse des Strings dem exec Systemaufruf als Parameter übergeben.
Konkret bedeutet das

\section{Format-String-Attacke}

\section{Return to Libc}

\section{Use after Free}

\chapter{Gegenmaßnahmen}

\section{NX-Bit}

\subsection{Angriffe NX-Bit}

\section{ASLR}

\subsection{KASLR}

\subsection{KARL}
%https://heise.de/-3767821

\subsection{KAISER/KPTI (Kernel- und User-Space "page table splitting")}
%https://heise.de/-3931562
%https://glm.io/131938

\subsection{PIE-Flag}

\subsection{Angriffe ASLR}

\subsubsection{Heap-Spraying}

\subsubsection{Double-Page-Fault}

\subsubsection{Sidechannel-Angriff}

\section{Stack-Cookie}

\section{Shadow Stack}

\section{Non Excutable Stack}

\section{Canary}

\section{CFI}

\section{Stack Smashing Protector (ehemals ProPolice)}

\section{Stack Guard}

\section{Syscall-Filtering (https://en.wikipedia.org/wiki/Seccomp)}

\section{Sandboxing/Container (Chroot, LXC, Jails, Apps, Browser, Java, ...)}

\subsection{Jailbreak}

\chapter{Finden von Overflows}

\chapter{Vermeiden von Overflows}

\end{document}
